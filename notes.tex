\documentclass[a4paper, titlepage]{report}

\usepackage[a4paper, margin=1in]{geometry}

\usepackage{amssymb}
\usepackage{external/bussproofs}


\begin{document}

\newcommand{\RULE}[3]{
  % #1 ``natural deduction'' or ``sequent``
  % #2 name of rule (e.g. ''axiom``)
  % #3 page number
  {
  \iffalse
    TODO FIXME automate adding to rules sections
  \fi
  \bigskip
  \textbf{\uppercase{#1 rule (#2 rule)}} - page #3 \\
  \indent
  }
}

\newcommand{\EXERCISE}[3]{
  % #1 exercise number
  % #2 page number
  % #3 exercise summary
  {
  \iffalse
    TODO FIXME automate adding to exercises list?
  \fi
  \bigskip
  Exercise #1 - page #2 \\
  \indent #3 \\
  \indent
  }
}

\title {WIP Notes from \\ Mathematical Logic \\ Oxford texts in logic 3 \\ Ian Chrisell \& Wilfred Hodges}
\author {Chris J. Hall}
\maketitle


\tableofcontents

\chapter{introduction}

This collection of notes is considered a Work In Progress
The current up-to-date version of the notes can be found at https://github.com/mkfifo/mathematical-logic-notes


The text covered by these notes is ``Mathematical Logic'' by Ian Chriswell and Wilfred Hodges,
Published by Oxford logic, Oxfored texts in logic 3.

The pusposes of these notes is in no way to serve as a substitute for the text.

When reading through the text I was often annoyed at the lack of complete solutions - many are ommitted.


I am writing these notes as a collection of all my fully worked answers, along with some minor additions that I found useful
along the way.


\chapter{Natural deduction logic}


Definition 2.1.2 - page 7
  A sequent is an expression of the form

  $(\Gamma \vdash \psi)$

  where $\psi$ is a statement (the conclusion of the sequent) and $\Gamma$ is a set of the statements (the assumptions of the sequent).

  We can read this as  ``$\Gamma entails \psi$'' (``gamma entails psi'')

(2.2) The sequent $(\Gamma \vdash \psi)$ means
  there is a proof whose conclusion is $\psi$ and whose undischarged assumptions are all in the set $\Gamma$.

When (2.2) is correct we say that the sequent is correct.

The set $\Gamma$ can be empty in which case we write $(\vdash \psi)$ (read ``turnstyle psi'')
this sequent is correct if and only if there is a proof of $\psi$ with no undischarged assumptions.

\bigskip

\RULE{sequent}{axiom}{8}
  If $\psi \subset \Gamma$ then the sequent $(\Gamma \vdash \psi)$ is correct.

\RULE{sequent}{transitive}{8}
  If $(\Delta \vdash \psi)$ is correct,
  and for every $\delta$ in $\Delta$ $(\Gamma \vdash \delta)$ is correct,
  then $(\Gamma \vdash \psi)$ is correct.

\RULE{natural deduction}{axiom}{8}
  Let $\phi$ be a statement, then
    $\phi$
  is a derivation, its conclusion $\phi$, and its undischarged assumptions $\phi$.



\chapter{Propositional logic}

hello world


\chapter{natural logic rules}
A collection of the naturak logic rules introduced in the text, and the place to find them

\chapter{sequent rules}
A collection of the sequent rules introduced in the text, and the place to find them

\chapter{syntax examples}
This section is to keep notes for myself on how to style latex

\begin{prooftree}
 \AxiomC{[D]}
 \noLine
 \UnaryInfC{$\phi$}
 
 \AxiomC{[D']}
 \noLine
 \UnaryInfC{$\psi$}
 
 \RightLabel{$(\wedge I)$}
 \BinaryInfC{$(\phi \wedge \psi)$}
 
 \AxiomC{[D'']}
 \noLine
 \UnaryInfC{$\chi$}
 
 \RightLabel{$(\wedge I)$}
 \BinaryInfC{$((\phi \wedge \psi) \wedge \chi)$}

\end{prooftree}

\bigskip

\begin{tabular}{c}
  $P_1$    \\
  $P_2$    \\
  ...      \\
  $P_{12}$ \\
  ...      \\
  $P_N$    \\
\end{tabular}

\bigskip

\begin{tabular} {c | c c}
 x & 1 & 2  \\ \hline
 4 & 4 & 8  \\
 5 & 5 & 10 \\
\end{tabular}

\bigskip

\begin{tabular} {c | c c}
 $\wedge$ & T & F \\
    \hline
 T  & T & F \\
 F  & F & F \\
\end{tabular}

\bigskip


\end{document}