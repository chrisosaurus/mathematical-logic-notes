\documentclass[a4paper, titlepage]{report}

\usepackage{hyperref}

\usepackage[a4paper, margin=1in]{geometry}

\usepackage{external/amssymb}
\usepackage{external/bussproofs}
\usepackage{external/cancel}

% http://tex.stackexchange.com/questions/17877/how-to-show-subsections-and-subsubsections-in-toc
% show chapter, section, subsection, subsubsection, and paragraph in TOC
\setcounter{tocdepth}{2}
% number chapter, and section (but not subsection) throughout text and in TOC
\setcounter{secnumdepth}{1}

% changing subsection formatting to not include trailing line-break
% http://tex.stackexchange.com/questions/22576/redefining-sectioning-commands
% vim `kpsewhich report.cls`
% shows
%\newcommand\subsection{\@startsection{subsection}{2}{\z@}%
%                                     {-3.25ex\@plus -1ex \@minus -.2ex}%
%                                     {1.5ex \@plus .2ex}%
%                                     {\normalfont\large\bfseries}}
%...
%\newcommand\paragraph{\@startsection{paragraph}{4}{\z@}%
%                                    {3.25ex \@plus1ex \@minus.2ex}%
%                                    {-1em}%
%                                    {\normalfont\normalsize\bfseries}}
% hybrid - same as report subsection but with negative section from paragraph to suppress trailing newline
\makeatletter
\renewcommand\subsection{\@startsection{subsection}{2}{\z@}
                                     {-3.25ex\@plus -1ex \@minus -.2ex}%
                                     {-1em}% only part changes from subsection - taken from paragraph
                                     {\normalfont\large\bfseries}}
\makeatother

\begin{document}

\newcommand{\RULE}[4]{
  % #1 ``natural deduction'' or ``sequent``
  % #2 name of rule (e.g. ''axiom``)
  % #3 page number
  % #4 body of rule
  {
  \iffalse
    TODO FIXME automate adding to rules sections
  \fi
  \bigskip
  \subsection{
    \texorpdfstring{
      \uppercase{#1 rule (#2 rule)}
    }{
      #1 rule (#2 rule)
    }
  }
  % attach page number to same line at far right side
  \hfill page #3 \\
  \indent
  #4
  }
}

\newcommand{\EXERCISE}[4]{
  % #1 exercise number
  % #2 page number
  % #3 exercise summary
  % #4 exercise solution
  {
  \iffalse
    TODO FIXME automate adding to exercises list?
  \fi
  \bigskip
  \subsection{
    \texorpdfstring{
      Exercise #1
    }{
      Exercise #1
    }
  }
  % attach page number to same line at far right side
  \hfill page #2 \\
  \indent
  #3 \\
  \indent
  #4
  }
}

% alpha page numbering for title page
% to suppress warning
\pagenumbering{alph}

\begin{titlepage}
\title {WIP Notes from \\ Mathematical Logic \\ Oxford texts in logic 3 \\ Ian Chrisell \& Wilfred Hodges}
\author {Chris J. Hall}
\maketitle
\end{titlepage}

% roman page numbering
\pagenumbering{roman}
\tableofcontents


% start page number 1 here
\newpage
\pagenumbering{arabic}

\chapter{introduction}


This collection of notes is considered a Work In Progress
The current up-to-date version of the notes can be found at \url{https://github.com/mkfifo/mathematical-logic-notes}


The text covered by these notes is ``Mathematical Logic'' by Ian Chriswell and Wilfred Hodges,
Published by Oxford logic, Oxfored texts in logic 3.

The pusposes of these notes is in no way to serve as a substitute for the text.

When reading through the text I was often annoyed at the lack of complete solutions - many are ommitted.


I am writing these notes as a collection of all my fully worked answers, along with some minor additions that I found useful
along the way.

There may be some exercises missing, some of the earlier examples seemed too trivial / boring - these should be very few
and any PRs to include them are welcome.


\chapter{Natural deduction logic}

\section{proofs and sequents}

Definition 2.1.2 - page 7
  A sequent is an expression of the form

  $(\Gamma \vdash \psi)$

  where $\psi$ is a statement (the conclusion of the sequent) and $\Gamma$ is a set of the statements (the assumptions of the sequent).

  We can read this as  ``$\Gamma entails \psi$'' (``gamma entails psi'')

(2.2) The sequent $(\Gamma \vdash \psi)$ means
  there is a proof whose conclusion is $\psi$ and whose undischarged assumptions are all in the set $\Gamma$.

When (2.2) is correct we say that the sequent is correct.

The set $\Gamma$ can be empty in which case we write $(\vdash \psi)$ (read ``turnstyle psi'')
this sequent is correct if and only if there is a proof of $\psi$ with no undischarged assumptions.

\bigskip

\RULE{sequent}{axiom}{8}{
  If $\psi \subset \Gamma$ then the sequent $(\Gamma \vdash \psi)$ is correct.
}
  
\RULE{sequent}{transitive}{8}{
  If $(\Delta \vdash \psi)$ is correct,
  and for every $\delta$ in $\Delta$ $(\Gamma \vdash \delta)$ is correct,
  then $(\Gamma \vdash \psi)$ is correct.
}
  
\RULE{natural deduction}{axiom}{8}{
  Let $\phi$ be a statement, then
    $\phi$
  is a derivation, its conclusion $\phi$, and its undischarged assumptions $\phi$.
}

\EXERCISE{2.1.3 A}{9}{
  If the sequent $(\Gamma \vdash \psi)$ is a correct sequent,
  and every statement in $\Gamma$ is also in $\Delta$,
  then the sequent $(\Delta \vdash \phi)$ is also correct.
}{
  Correct.
  
  If $\Delta$ is a superset of $\Gamma$ (as in this question) then we can
  always build $\psi$ from it.
  
  if $(\Gamma \vdash \psi)$ and $(\Delta \subseteq \Gamma)$ then $(\Delta \vdash \psi)$.
}

\EXERCISE{2.1.3 B}{9}{
  If the sequent $(\{\psi\} \vdash \psi)$ is correct,
  then so is the sequent $(\{\psi\} \vdash \phi)$.
}{
  ??? Correct.
  
  Reasoning goes here.
}

\EXERCISE{2.1.3 C}{9}{
  If $(\Gamma \vdash \psi)$ and $(\Delta \vdash \psi)$ are correct,
  then so is $((\Gamma \cap \Delta) \vdash \psi)$.
}{
  Incorrect.
  
  Here we use the $\cap$ or 'intersection' operator,
  this will only yield the set of statements in both of the sets.
  If there is no overlap, then this will yield the empty set ($\varnothing$).
  
  For example if we have $( \{ ``x = 1'' \} \vdash ``x > 0'' )$ and $( \{ ``x = 2'' \} \vdash ``x > 0'' )$, both of which are well formed and correct.
  
  However $( (\{ ``x = 1'' \} \cap \{ ``x = 2'' \}) \vdash ``x > 0'' )$ is incorrect,
  as the value of $(\{ ``x = 1'' \} \cap \{ ``x = 2'' \})$ is $\varnothing$ and $( \{ \varnothing \} \vdash ``x > 0'' )$ is incorect.
  
  Notice however that this would be true:
    If $(\Gamma \vdash \psi)$ and $(\Delta \vdash \psi)$ are correct,
    then so is $((\Gamma \cup \Delta) \vdash \psi)$.
  as we are using the 'union' ($\cup$) operator
  which will construct a set containing all the elements from both sets
  so the value of $(\{ ``x = 1'' \} \cup \{ ``x = 2'' \})$ is $\{ ``x = 1'', ``x = 2'' \}$ and $( \{ ``x = 1'', ``x = 2'' \} \vdash ``x > 0'' )$ is corect.
}


\chapter{Propositional logic}

hello world


\chapter{natural logic rules}
A collection of the naturak logic rules introduced in the text, and the place to find them

\chapter{sequent rules}
A collection of the sequent rules introduced in the text, and the place to find them

\chapter{syntax examples}
This section is to keep notes for myself on how to style latex

\url{https://oeis.org/wiki/List_of_LaTeX_mathematical_symbols}

$\cancel{\phi}$
  
\begin{prooftree}
 \AxiomC{[D]}
 \noLine
 \UnaryInfC{$\phi$}
 
 \AxiomC{[D']}
 \noLine
 \UnaryInfC{$\psi$}
 
 \RightLabel{$(\wedge I)$}
 \BinaryInfC{$(\phi \wedge \psi)$}
 
 \AxiomC{[D'']}
 \noLine
 \UnaryInfC{$\chi$}
 
 \RightLabel{$(\wedge I)$}
 \BinaryInfC{$((\phi \wedge \psi) \wedge \chi)$}

\end{prooftree}

\bigskip

\begin{tabular}{c}
  $P_1$    \\
  $P_2$    \\
  ...      \\
  $P_{12}$ \\
  ...      \\
  $P_N$    \\
\end{tabular}

\bigskip

\begin{tabular} {c | c c}
 x & 1 & 2  \\ \hline
 4 & 4 & 8  \\
 5 & 5 & 10 \\
\end{tabular}

\bigskip

\begin{tabular} {c | c c}
 $\wedge$ & T & F \\
    \hline
 T  & T & F \\
 F  & F & F \\
\end{tabular}

\bigskip


\end{document}
